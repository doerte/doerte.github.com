%%%%%%%%%%%%%%%%%%%%%%%%%%%%%%%%%%%%%%%%%%%%%%%%%%%%%%%%%%%%%%%%%%%%%%%%
%%%%%%%%%%%%%%%%%%%%%% Simple LaTeX CV Template %%%%%%%%%%%%%%%%%%%%%%%%
%%%%%%%%%%%%%%%%%%%%%%%%%%%%%%%%%%%%%%%%%%%%%%%%%%%%%%%%%%%%%%%%%%%%%%%%

%%%%%%%%%%%%%%%%%%%%%%%%%%%%%%%%%%%%%%%%%%%%%%%%%%%%%%%%%%%%%%%%%%%%%%%%
%% NOTE: If you find that it says                                     %%
%%                                                                    %%
%%                           1 of ??                                  %%
%%                                                                    %%
%% at the bottom of your first page, this means that the AUX file     %%
%% was not available when you ran LaTeX on this source. Simply RERUN  %%
%% LaTeX to get the ``??'' replaced with the number of the last page  %%
%% of the document. The AUX file will be generated on the first run   %%
%% of LaTeX and used on the second run to fill in all of the          %%
%% references.                                                        %%
%%%%%%%%%%%%%%%%%%%%%%%%%%%%%%%%%%%%%%%%%%%%%%%%%%%%%%%%%%%%%%%%%%%%%%%%

%%%%%%%%%%%%%%%%%%%%%%%%%%%% Document Setup %%%%%%%%%%%%%%%%%%%%%%%%%%%%

% Don't like 10pt? Try 11pt or 12pt
\documentclass[11pt]{article}

% This is a helpful package that puts math inside length specifications
\usepackage{calc}
\usepackage{marvosym}
\usepackage[nodayofweek]{datetime}
\usepackage{graphicx}

% Simpler bibsection for CV sections
% (thanks to natbib for inspiration)
\makeatletter
\newlength{\bibhang}
\setlength{\bibhang}{1em}
\newlength{\bibsep}
 {\@listi \global\bibsep\itemsep \global\advance\bibsep by\parsep}
\newenvironment{bibsection}%
        {\vspace{-\baselineskip}\begin{list}{}{%
       \setlength{\leftmargin}{\bibhang}%
       \setlength{\itemindent}{-\leftmargin}%
       \setlength{\itemsep}{\bibsep}%
       \setlength{\parsep}{\z@}%
        \setlength{\partopsep}{0pt}%
        \setlength{\topsep}{0pt}}}
        {\end{list}\vspace{-.6\baselineskip}}
\makeatother

% Layout: Puts the section titles on left side of page
\reversemarginpar

%
%         PAPER SIZE, PAGE NUMBER, AND DOCUMENT LAYOUT NOTES:
%
% The next \usepackage line changes the layout for CV style section
% headings as marginal notes. It also sets up the paper size as either
% letter or A4. By default, letter was used. If A4 paper is desired,
% comment out the letterpaper lines and uncomment the a4paper lines.
%
% As you can see, the margin widths and section title widths can be
% easily adjusted.
%
% ALSO: Notice that the includefoot option can be commented OUT in order
% to put the PAGE NUMBER *IN* the bottom margin. This will make the
% effective text area larger.
%
% IF YOU WISH TO REMOVE THE ``of LASTPAGE'' next to each page number,
% see the note about the +LP and -LP lines below. Comment out the +LP
% and uncomment the -LP.
%
% IF YOU WISH TO REMOVE PAGE NUMBERS, be sure that the includefoot line
% is uncommented and ALSO uncomment the \pagestyle{empty} a few lines
% below.
%

%% Use these lines for letter-sized paper
%\usepackage[paper=letterpaper,
%            %includefoot, % Uncomment to put page number above margin
%            marginparwidth=1.2in,     % Length of section titles
%            marginparsep=.05in,       % Space between titles and text
%            margin=1in,               % 1 inch margins
%            includemp]{geometry}

% Use these lines for A4-sized paper
\usepackage[paper=a4paper,
            %includefoot, % Uncomment to put page number above margin
            marginparwidth=30.5 mm,    % Length of section titles
            marginparsep=1.5mm,       % Space between titles and text
            margin=25mm,              % 25mm margins
            includemp]{geometry}

%% More layout: Get rid of indenting throughout entire document
\setlength{\parindent}{0in}

%% This gives us fun enumeration environments. compactitem will be nice.
\usepackage{paralist}

%% Reference the last page in the page number
%
% NOTE: comment the +LP line and uncomment the -LP line to have page
%       numbers without the ``of ##'' last page reference)
%
% NOTE: uncomment the \pagestyle{empty} line to get rid of all page
%       numbers (make sure includefoot is commented out above)
%
\usepackage{fancyhdr,lastpage}
\pagestyle{fancy}
%\pagestyle{empty}      % Uncomment this to get rid of page numbers
\fancyhf{}\renewcommand{\headrulewidth}{0pt}
\fancyfootoffset{\marginparsep+\marginparwidth}
\newlength{\footpageshift}
\setlength{\footpageshift}
          {0.5\textwidth+0.5\marginparsep+0.5\marginparwidth-2in}
\lfoot{\hspace{\footpageshift}%
       \parbox{4in}{\, \hfill %
                    \arabic{page} of 4 % +LP
%                    \arabic{page}                               % -LP
                    \hfill \,}}

% Finally, give us PDF bookmarks
\usepackage{color,hyperref}
\definecolor{darkblue}{rgb}{0.0,0.0,0.3}
\hypersetup{colorlinks,breaklinks,
            linkcolor=darkblue,urlcolor=darkblue,
            anchorcolor=darkblue,citecolor=darkblue}

%%%%%%%%%%%%%%%%%%%%%%%% End Document Setup %%%%%%%%%%%%%%%%%%%%%%%%%%%%


%%%%%%%%%%%%%%%%%%%%%%%%%%% Helper Commands %%%%%%%%%%%%%%%%%%%%%%%%%%%%

% The title (name) with a horizontal rule under it
% (optional argument typesets an object right-justified across from name
%  as well)
%
% Usage: \makeheading{name}
%        OR
%        \makeheading[right_object]{name}
%
% Place at top of document. It should be the first thing.
% If ``right_object'' is provided in the square-braced optional
% argument, it will be right justified on the same line as ``name'' at
% the top of the CV. For example:
%
%       \makeheading[\emph{Curriculum vitae}]{Your Name}
%
% will put an emphasized ``Curriculum vitae'' at the top of the document
% as a title. Likewise, a picture could be included:
%
%   \makeheading[\includegraphics[height=1.5in]{my_picutre}]{Your Name}
%
% the picture will be flush right across from the name.
\newcommand{\makeheading}[2][]%
        {\hspace*{-\marginparsep minus \marginparwidth}%
         \begin{minipage}[t]{\textwidth+\marginparwidth+\marginparsep}%
             {\large \bfseries #2 \hfill #1}\\[-0.15\baselineskip]%
                 \rule{\columnwidth}{1pt}%
         \end{minipage}}

% The section headings
%
% Usage: \section{section name}
%
% Follow this section IMMEDIATELY with the first line of the section
% text. Do not put whitespace in between. That is, do this:
%
%       \section{My Information}
%       Here is my information.
%
% and NOT this:
%
%       \section{My Information}
%
%       Here is my information.
%
% Otherwise the top of the section header will not line up with the top
% of the section. Of course, using a single comment character (%) on
% empty lines allows for the function of the first example with the
% readability of the second example.
\renewcommand{\section}[2]%
        {\pagebreak[3]\vspace{1.3\baselineskip}%
         \phantomsection\addcontentsline{toc}{section}{#1}%
         \hspace{0in}%
         \marginpar{
         \raggedright \scshape #1}#2}

% An itemize-style list with lots of space between items
\newenvironment{outerlist}[1][\enskip\textbullet]%
        {\begin{itemize}[#1]}{\end{itemize}%
         \vspace{-.6\baselineskip}}

% An environment IDENTICAL to outerlist that has better pre-list spacing
% when used as the first thing in a \section
\newenvironment{lonelist}[1][\enskip\textbullet]%
        {\vspace{-\baselineskip}\begin{list}{#1}{%
        \setlength{\partopsep}{0pt}%
        \setlength{\topsep}{0pt}}}
        {\end{list}\vspace{-.6\baselineskip}}

% An itemize-style list with little space between items
\newenvironment{innerlist}[1][\enskip\textbullet]%
        {\begin{compactitem}[#1]}{\end{compactitem}}

% An environment IDENTICAL to innerlist that has better pre-list spacing
% when used as the first thing in a \section
\newenvironment{loneinnerlist}[1][\enskip\textbullet]%
        {\vspace{-\baselineskip}\begin{compactitem}[#1]}
        {\end{compactitem}\vspace{-.6\baselineskip}}

% To add some paragraph space between lines.
% This also tells LaTeX to preferably break a page on one of these gaps
% if there is a needed pagebreak nearby.
\newcommand{\blankline}{\quad\pagebreak[3]}
\newcommand{\halfblankline}{\quad\vspace{-0.5\baselineskip}\pagebreak[3]}

% Uses hyperref to link DOI
\newcommand\doilink[1]{\href{http://dx.doi.org/#1}{#1}}
\newcommand\doi[1]{doi:\doilink{#1}}

% For \url{SOME_URL}, links SOME_URL to the url SOME_URL
\providecommand*\url[1]{\href{#1}{#1}}
% Same as above, but pretty-prints SOME_URL in teletype fixed-width font
\renewcommand*\url[1]{\href{#1}{\texttt{#1}}}

% For \email{ADDRESS}, links ADDRESS to the url mailto:ADDRESS
\providecommand*\email[1]{\href{mailto:#1}{#1}}
% Same as above, but pretty-prints ADDRESS in teletype fixed-width font
%\renewcommand*\email[1]{\href{mailto:#1}{\texttt{#1}}}

%\providecommand\BibTeX{{\rm B\kern-.05em{\sc i\kern-.025em b}\kern-.08em
%    T\kern-.1667em\lower.7ex\hbox{E}\kern-.125emX}}
%\providecommand\BibTeX{{\rm B\kern-.05em{\sc i\kern-.025em b}\kern-.08em
%    \TeX}}
\providecommand\BibTeX{{B\kern-.05em{\sc i\kern-.025em b}\kern-.08em
    \TeX}}
\providecommand\Matlab{\textsc{Matlab}}

%%%%%%%%%%%%%%%%%%%%%%%% End Helper Commands %%%%%%%%%%%%%%%%%%%%%%%%%%%

\newenvironment{fullwidth}{\par\leftskip=-30.5mm \rightskip=0cm}{\par}

%%%%%%%%%%%%%%%%%%%%%%%%% Begin CV Document %%%%%%%%%%%%%%%%%%%%%%%%%%%%
\begin{document}
\thispagestyle{empty}


\makeheading[\emph{Curriculum Vitae}]{D\"{o}rte A. Hessler}

\section{Contact Information}
%
% NOTE: Mind where the & separators and \\ breaks are in the following
%       table.
%
% ALSO: \rcollength is the width of the right column of the table
%       (adjust it to your liking; default is 1.85in).
%
\newlength{\rcollength}\setlength{\rcollength}{2in}%
%
\begin{tabular}[t]{l l}

\textit{E-mail:} & \email{me@doerte.eu}\\
\textit{WWW:} &\href{http://doerte.eu/}{http://doerte.eu/}\\
\end{tabular}



\section{Date and Place of Birth}
%
14 October 1981, Cuxhaven, Germany

\section{Citizenship}
% 
German

\section{Work Experience}
%

\textbf{University Lecturer}  \hfill September 2011 to present
\begin{outerlist}
	\item[] Department of Neurolinguistics, University of Groningen, The Netherlands
	\item[] Teaching on Bachelor level in the department of linguistics (0.2 fte)\\
	
\end{outerlist}

\textbf{Owner translation agency} \hfill August 2011 to present
\begin{outerlist}
	
	\item[] Translations Dutch-German, interpretations Dutch-German, German-Dutch and corrections of German texts
	\item[] More information on the company website \href{http://parolanto.nl}
             {http://parolanto.nl}
	
\end{outerlist}


\section{Education}
%
\textbf{University of Groningen, The Netherlands} \hfill September 2007 to September 2011
\begin{outerlist}
\item[]Ph.D.,
        \href{http://www.let.rug.nl/neurolinguistics/}
             {Neurolinguistics},
             planned defense: 15 December 2011
        \begin{innerlist}
        \item Thesis Title: \emph{Auditory and Audiovisual Processing in Aphasic and Non-Brain-Damaged Listeners: The Whole is More than the Sum of its Parts}
        \item Advisors:
 		\begin{itemize}             
                   \item[]Prof. Roelien Bastiaanse (Promotor)
		\item[]Dr. Roel Jonkers (Co-promotor)
        		\end{itemize}
	\item Graduate School: School of Behavioral and Cognitive Neuroscience (BCN), Groningen\\
        \end{innerlist}
\end{outerlist}

\textbf{University of Potsdam, Germany}  \hfill  October 2001 to May 2007
\begin{outerlist}

\item[]Diplom (cum laude),
        \href{http://www.ling.uni-potsdam.de/patho/index.html}
             {General Linguistics (Patholinguistics)}, May 2007
        \begin{innerlist}
        \item Thesis Topic: \emph{War die st\"{o}rungsspezifische Behandlung der 
	auditiven Analyse effektiv? Eine Einzelfallstudie bei Aphasie\\ 
	(Was the specific treatment of the auditory analysis effective? A single case study on aphasia)}
        \item Adviser: Dr. Nicole Stadie
        \item Areas of Study: Neuro- and Psycholinguistics\\
        \end{innerlist}
\end{outerlist}

\textbf{Highschool education}  \hfill September 1998 to July 2001
\begin{outerlist}

\item[]Abitur 
        \begin{innerlist}
        \item Schulzentrum Geschwister Scholl, Bremerhaven, Germany
	\item August 1999 to June 2001
        \end{innerlist}

\item[]Exchange student
	\begin{innerlist}
	\item Westwood High School, Fort Pierce, Florida, USA
	\item September 1998 to July 1999
	\end{innerlist}
\end{outerlist}


\section{Publications} \begin{bibsection}
 \item Hessler, D, Jonkers, R. and Bastiaanse, R. (in press). Processing of audiovisual stimuli in aphasic and non-brain-damaged listeners. \emph{Aphasiology}. 

  \item Hessler, D. (in press). Audiovisuelle Verarbeitung von Phonemen bei Aphasie. 
      In Hanne, S., Fritzsche, T., Ott, S., Adelt, A. (eds.) \emph{Spektrum Patholinguistik} (4). 		          
	Potsdam: Universitätsverlag Potsdam.

  \item Hessler, D., Jonkers, R. and Bastiaanse, R. (2010). The influence of phonetic dimensions on aphasic speech perception. \emph{Clinical Linguistics and Phonetics} 24, 980-996.

\item Hessler, D. and Stadie, N. (2008). War die st\"{o}rungsspezifische Behandlung der auditiven Analyse effektiv? Eine Einzelfallstudie bei Aphasie. In Wahl, M., Heide, J., Hanne, S. (eds.) \emph{Spektrum Patholinguistik} (1). Potsdam: Universitätsverlag Potsdam, 131-134.
  
\end{bibsection}

\section{Submitted\\ for\\ Publication} \begin{bibsection}
    \item Hessler, D., Jonker, R., Stowe, L. and Bastiaanse, R. (submitted). The whole is more than the sum of its parts -
audiovisual processing of phonemes investigated with
ERPs. \emph{Brain and Language}.
\end{bibsection}

% Add a little space to nudge next ``Conference Publications'' marginpar
% down to make room for tall ``Submitted Journal Publications''
% marginpar. If there are enough submitted journal publications, this
% space will not be needed (and should be removed).
\vspace{0.1in}

\section{Presentations}
\begin{bibsection}
\item Hessler, D., Jonkers, R., Stowe, L. and Bastiaanse, R. (2011). Processing of phonemic contrasts - an ERP study. Talk given at `Science of Aphasia XII', Barcelona (Spain). 

\item Hessler, D., Jonkers, R., Stowe, L. and Bastiaanse, R. (2011). The psychological reality of phonemic contrasts. Talk given at the `32\textsuperscript{nd} TABU Dag', Groningen (The Netherlands). 

\item Hessler, D., Jonkers, R. and Bastiaanse, R. (2011). Processing of audiovisual stimuli in aphasic and non-brain-damaged listeners. Poster presented at `BCN New Years Meeting', Groningen (Netherlands).

\item Hessler, D. (2010). Audiovisuelle Verarbeitung von Phonemen bei Aphasie. Invited talk given at `4. Herbsttreffen Patholinguistik', Potsdam (Germany).

\item Hessler, D., Jonkers, R. and Bastiaanse, R. (2010). Deviant processing of audiovisual stimuli in aphasia. Talk given at `Science of Aphasia XI', Potsdam (Germany). 

\item Hessler, D., Jonkers, R. and Bastiaanse, R. (2010). Audiovisual processing in aphasic and healthy subjects - differences revealed by reaction times. Talk given at the `31\textsuperscript{st} TABU Dag', Groningen (The Netherlands). 

\item Hessler, D., Jonkers, R. and Bastiaanse, R. (2009).The influence of phonetic features on aphasic speech perception. Talk given at `Science of Aphasia X', Antalya (Turkey). 

\item Hessler, D., Jonkers, R. and Bastiaanse, R. (2009). How aphasic listeners perceive different phonetic features. Poster presented at `The 3\textsuperscript{rd} International Conference on Auditory Cortex', Magdeburg (Germany).

\item Hessler, D. and Stadie, N. (2008). Evaluation of treatment for word sound deafness in aphasia - A single case study. Talk given at `Science of Aphasia IX', Chalkidiki (Greece). 

\item Hessler, D. and Stadie, N. (2007). War die st\"{o}rungsspezifische Behandlung der auditiven Analyse effektiv? Eine Einzelfallstudie bei Aphasie. Poster presented at `1. Herbsttreffen Patholinguistik', Potsdam (Germany).

\item Hessler, D. and Stadie, N. (2006). Hat die st\"{o}rungsspezifische Behandlung der auditiven Analyse gewirkt? Eine Einzelfallstudie bei Aphasie. Poster presented at `GAB Conference', Hamburg (Germany).

\item Hessler, D. and Stadie, N. (2006). Evaluation of a treatment of word sound deafness in an aphasic patient. Talk given at `BAS Therapy Symposium', Plymouth (UK).

\end{bibsection}


\section{Teaching Experience}
\textbf{University of Groningen, The Netherlands}
    \begin{innerlist}
        
        \item Guest lecture for ``European Master in Clinical Linguistics"
        \begin{innerlist}
            \item Fall~2009 and Fall~2010
	  \item Topic: The influence of speechreading on (aphasic) comprehension
        \end{innerlist}

 	\halfblankline

        \item Guest lecture ``Hot topics in linguistics''
        \begin{innerlist}
            \item Spring~2010
	   \item Topic:  “Stoornissen van taalbegrip bij afasie en de invloed van liplezen erop” (``Deficits in language perception in aphasia and the influence of lip-reading on them'')

        \end{innerlist}

\halfblankline

        \item Guest Lecture ``Onderzoekscollege Neurolinguistiek''

        \begin{innerlist}
            \item Spring~2009
 	   \item Topic: The McGurk effect in Aphasia
        \end{innerlist}

        \halfblankline

        \item Guest lecture ``Hot topics in linguistics''
        \begin{innerlist}
            \item Spring~2009
	   \item Topic:  “Luisteren en kijken om te begrijpen - multimodaliteit van taalperceptie” (``Listening and watching to understand - multimodality of speech perception'')

        \end{innerlist}

      \halfblankline

       \item Practicum: Neurolinguistiek I (Aphasiology part)
        
	\begin{innerlist}
            \item Fall 2008 
            \item Responsible for 4 practicum sessions: Neuroanatomy, History of Aphasia, Paraphasias and Social Impacts of Aphasia
        \end{innerlist}

  

\end{innerlist}





\section{Professional Activities}
%
Co-founder and chair of the PhD Student Council of the Graduate School of Humanities (University of Groningen): January 2010 to January 2011\\

Representative of PhD-students in the CLCG Advisory Council (University of Groningen): January 2010 to February 2011\\

Co-organizer of the \href{http://www.let.rug.nl/tabudag/archive/2009/index.php}{30\textsuperscript{th} TABU Dag}, an annual linguistic conference in Groningen: June 2009 \\

Organization of the CLCG Graduate Student Meeting (University of Groningen): September 2008, September 2009 and September 2010\\

Secretary of the Neurolinguistics Research Group (University of Groningen): September 2008 to January 2010\\

Webcoordinator of the Neurolinguistics Research Group (University of Groningen): September 2007 to present

\section{Computer Skills}
%
\textbf{Operating systems}
\begin{innerlist}
\item Windows 7, Vista, XP, NT, 2000, 98 
\item MacOS X - basic knowledge
\item Linux (Debian, Ubuntu)
\end{innerlist}                                 

\halfblankline

\textbf{Office software and text processing}
\begin{innerlist}
\item MS Office 2003-2011 (Windows and Mac editions)
\item \LaTeX
\end{innerlist}

\halfblankline

\textbf{Statistics software}
\begin{innerlist}
\item SPSS
\item Statistica
\item R
\end{innerlist}

\halfblankline

\textbf{Experimental software}
\begin{innerlist}
\item E-Prime
\item DMDX
\item Childes/CLAN
\item Brain Vision Recorder
\item Brain Vision Analyzer
\item Matlab - basic knowledge
\end{innerlist}

\halfblankline

\textbf{Programming languages}
\begin{innerlist}
\item Ruby - basic knowledge
\item Matlab - basic knowledge
\end{innerlist}

\halfblankline

\textbf{Photo- and videoediting software}
\begin{innerlist}
\item Adobe Creative Suite 4 and 5
\item Adobe Premiere
\end{innerlist}

\section{Languages}
%
\textbf{German}: native language\\
\textbf{Dutch}: fluent in oral and written modality (Diploma NT2 II)\\
\textbf{English}: fluent in oral and written modality

\section{Hobbies and interests}
%
Singing, indoor soccer, Zumba, reading (novels), acquiring new computer skills (e.g. learning to program)
\newpage






\end{document}

%%%%%%%%%%%%%%%%%%%%%%%%%% End CV Document %%%%%%%%%%%%%%%%%%%%%%%%%%%%%

%----------------------------------------------------------------------%
% The following is copyright and licensing information for
% redistribution of this LaTeX source code; it also includes a liability
% statement. If this source code is not being redistributed to others,
% it may be omitted. It has no effect on the function of the above code.
%----------------------------------------------------------------------%
% Copyright (c) 2007, 2008, 2009, 2010, 2011 by Theodore P. Pavlic
%
% Unless otherwise expressly stated, this work is licensed under the
% Creative Commons Attribution-Noncommercial 3.0 United States License. To
% view a copy of this license, visit
% http://creativecommons.org/licenses/by-nc/3.0/us/ or send a letter to
% Creative Commons, 171 Second Street, Suite 300, San Francisco,
% California, 94105, USA.
%
% THE SOFTWARE IS PROVIDED "AS IS", WITHOUT WARRANTY OF ANY KIND, EXPRESS
% OR IMPLIED, INCLUDING BUT NOT LIMITED TO THE WARRANTIES OF
% MERCHANTABILITY, FITNESS FOR A PARTICULAR PURPOSE AND NONINFRINGEMENT.
% IN NO EVENT SHALL THE AUTHORS OR COPYRIGHT HOLDERS BE LIABLE FOR ANY
% CLAIM, DAMAGES OR OTHER LIABILITY, WHETHER IN AN ACTION OF CONTRACT,
% TORT OR OTHERWISE, ARISING FROM, OUT OF OR IN CONNECTION WITH THE
% SOFTWARE OR THE USE OR OTHER DEALINGS IN THE SOFTWARE.
%----------------------------------------------------------------------%
